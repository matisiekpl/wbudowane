\documentclass{article}
\usepackage{graphicx}
\usepackage[T1]{fontenc}
\usepackage[polish]{babel}
\usepackage[utf8]{inputenc}
\usepackage{listings}

\title{System do monitorowania poziomu glukozy we krwi}
\author{inż. Mateusz Woźniak}
\date{Grudzień 2025}

\begin{document}

    \maketitle


    \section{Wprowadzenie}

    Celem niniejszego projektu jest opracowanie urządzenia IoT do monitorowania poziomu glukozy we krwi.
    Założenia systemu:
    \begin{itemize}
        \item Centralnym komponentem urządzenia jest Arduino UNO
        \item Czujnikiem, który generuje sygnał o poziomie glukozy we krwi jest potencjometr (w rzeczywistości jest to prawdziwy czujnik)
        \item Urządzenie posiada wyświetlacz LCD do pokazywania bieżących i historycznych odczytów
        \item Pomiary są archiwizowane na karcie SD w pliku CSV.
        \item Buzer informuje o krytycznym poziomie glukozy we krwi.
        \item Przycisk powoduje przełączanie się między trybami wyświetlania w LCD (bieżące/archiwalne).
    \end{itemize}


    \section{Realizacja projektu}

    \subsection{Schemat połączenia obwodu}
    Używając symulatora Wokwi utworzono schemat połączeń komponentów takich jak Arduino UNO, Buzzer, czytnik kard SD, wyświetlacz, potencjometr i zegar RTC.

    \begin{figure}
        \centering
        \includegraphics[width=0.9\textwidth]{circuit}\label{fig:schemat}
        \caption{Schemat połączeń układu}
    \end{figure}

    Schemat przedstawiono na rys.~\ref{fig:schemat}.

    Poniżej przedstawiono użyte protokoły komunikacji:
    \begin{itemize}
        \item Zegar RTC jest podłączony z użyciem protokołu I2C.
        \item Wyświetlacz LCD jest podłączony z użyciem protokołu I2C.
        \item Czytnik kart SD jest podłączony z użyciem protokołu SPI.
        \item Potencjometr jest podłączony za pomocą wejścia anaologowego.
        \item Buzzer jest podłączony za pomocą wyjścia PWM.
        \item Przełącznik bistabilny jest podłączony za pomocą wejścia cyfrowego.
    \end{itemize}

    \subsection{Implementacja oprogramowania}

    Oprogramowanie zostało napisane w środowisku C/Arduino.

    W programie istnieje pętla główna, która jest wykonywana z częstotliwością 100 Hz. Ma ona za zadanie reagować na sygnały wejścia oraz aktualizować wyświetlacz LCD. Pętla ta również sprawdza, czy w danej chwili system powinien pomiar.
    Interwał pomiaru jest ustawiony na 1 sekundę.
    Jeżeli warunek czasowy wykonania pomiaru zostanie spełniony, urządzenie wykona odczyt z wejścia analogowego, a następnie wykona przetworzenie odczytu.
    W ramach przetwarzania urządzenie zapisze odczyt do pliku CSV wraz z aktualną pieczęcią czasu Unix, sprawdzi, czy pomiar powinien spowodować uruchomienie alarmu (buzzer) oraz wyświetli wynik na ekranie LCD.

    W przypadku, gdy użytkownik przełączy przycisk na tryb archiwalny, pętla główna będzie wyświetlać zagregowane pomiary na ekranie LCD (średnie czasowe).

    Logikę działania systemu przedstawiono na rys.~\ref{fig:logic}.

    \begin{figure}
        \centering
        \includegraphics[width=0.9\textwidth]{logic}\label{fig:logic}
        \caption{Diagram blokowy działania systemu}
    \end{figure}


    W listingu~\ref{code} przedstawiono kod źródłowy.

    \lstinputlisting[frame=single, caption={Kod źródłowy}, label=code, numbers=left]{code.c}


    \section{Działanie i testy systemu}
    Urządzenie było przede wszystkim testowane manualnie wraz z rozwojem funkcjonalności projektu.

    Stwierdzono, że urządzenie działa w pełni zgodnie z wyznaczonymi założeniami, a każda funkcjonalność działa bez błędów.

    Na załączonych zrzutach ekranu przedstawiono stan testowania funkcji.

    \begin{figure}
        \centering
        \includegraphics[width=0.9\textwidth]{demo1}\label{fig:demo1}
        \caption{Podstawowy tryb wyświetlania wraz z paskiem informacyjnym}
    \end{figure}

    \begin{figure}
        \centering
        \includegraphics[width=0.9\textwidth]{demo2}\label{fig:demo2}
        \caption{Urządzenie w stanie ostrzegawczym - buzzer "pika"}
    \end{figure}

    \begin{figure}
        \centering
        \includegraphics[width=0.9\textwidth]{demo3}\label{fig:demo3}
        \caption{Urządzenie w stanie krytycznym - buzzer wydaje dźwięk ciągły}
    \end{figure}

    \begin{figure}
        \centering
        \includegraphics[width=0.9\textwidth]{demo4}\label{fig:demo4}
        \caption{Urządzenie w trybie archiwalnym - LCD pokazuje średnie czasowe}
    \end{figure}


    \section{Wnioski}
    Opracowanie systemu do monitorowania poziomu glukozy we krwi stanowiło dobrą okazję do nauczenia się budowania kompletnych urządzeń wbudowanych.
    W szczególności trudne było dobre zaprojektowanie pętli, tak by odczyt był wykonywany z częstotliwości 1Hz, a jednocześnie wyświetlacz LCD był odświeżany z częstotliwością 100Hz - ten kluczowy element oprogramowania powinien zostać zaplanowany na samym początku implementacji kodu, a nie w trakcie.
    Pomimo małych przeszkód, projekt został wykonany w pełni, zgodnie z założeniami początkowymi.


    \section{Możliwe ścieżki rozwoju projektu}
    W trakcie budowy projektu pojawiły się pomysły potencjalnego rozszerzenia projektu o:
    \begin{itemize}
        \item Moduł Bluetooth HC-06 - umożliwiałby on strumieniowanie odczytów do innych komponentów np. smartfon
        \item Moduł GSM - umożliwiałby on strumieniowanie odczytów do sieci internetu komórkowego
    \end{itemize}

    Opisane powyżej ścieżki rozwoju mogłyby pozwolić na sprawne przekazywanie informacji lekarzowi w czasie rzeczywistym, co pozwoliłoby na jeszcze lepsze efekty leczenia chorób.


\end{document}

